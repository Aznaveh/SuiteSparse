\documentclass[12pt]{article}
\usepackage{hyperref}
\usepackage{minted}

\topmargin -0.5in
\textheight 9.0in
\oddsidemargin 0pt
\evensidemargin 0pt
\textwidth 6.5in

%-------------------------------------------------------------------------------
% get epsf.tex file, for encapsulated postscript files:
\input epsf
%-------------------------------------------------------------------------------
% macro for Postscript figures the easy way
% usage:  \postscript{file.ps}{scale}
% where scale is 1.0 for 100%, 0.5 for 50% reduction, etc.
%
\newcommand{\postscript}[2]
{\setlength{\epsfxsize}{#2\hsize}
\centerline{\epsfbox{#1}}}
%-------------------------------------------------------------------------------

\title{User's Guide for ParU, an unsymmetric multifrontal multithreaded sparse
LU factorization package}
\author{Mohsen Aznaveh\thanks{
email: aznaveh@tamu.edu.
http://www.suitesparse.com.
},
Timothy A. Davis}

\date{VERSION 0.0.0, May 20, 2022}

%-------------------------------------------------------------------------------
\begin{document}
%-------------------------------------------------------------------------------
\maketitle

\begin{abstract}

ParU is an implementation of the multifrontal sparse LU factorization
method.  Parallelism is exploited both in the BLAS and across different frontal
matrices using OpenMP tasking, a shared-memory programming model for modern 
multicore architectures. The package is written in C++ and real sparse matrices 
are supported.

\end{abstract}

\maketitle

%-------------------------------------------------------------------------------
\section{Introduction}
\label{intro}
%-------------------------------------------------------------------------------

The algorithms used in ParU will be discussed in a companion paper,
?. This document gives detailed information on the installation
and use of ParU.
ParU is a parallel sparse direct solver. This package uses OpenMP
tasking for parallelism. ParU calls UMFPACK for symbolic analysis phase,
after that some symbolic analysis is done by ParU itself and  then numeric
phase starts. The numeric computation is a task parallel phase using OpenMP
and each task calls parallel BLAS; i.e. nested parallism. 
The performance of BLAS has a heavy impact on the performance of ParU.
However, depending on the input problem performance of parallelsim in BLAS 
sometimes does not have effects in ParU.


%-------------------------------------------------------------------------------
\subsubsection{Instructions on using METIS}
%-------------------------------------------------------------------------------

SuiteSparse now on METIS 5.1.0, which is distributed along with
SuiteSparse itself.  Its use is optional, however. ParU is using METIS as the 
default ordering. METIS tends to give orderings that are good for the 
parallelism. You can compile and run your code without using METIS; We recommend 
using METIS along with ParU.

Note that METIS is not bug-free; it can occasionally cause segmentation 
faults, particularly if used when finding basic solutions to underdetermined 
systems with many more columns than rows .ParU do not solve such 
systems anyway but you might see some problems with other SuiteSparse packages.

%-------------------------------------------------------------------------------
\section{Using ParU in C and C++}
%-------------------------------------------------------------------------------

ParU relies on CHOLMOD for its basic sparse matrix data structure, a compressed 
sparse column format.  CHOLMOD provides interfaces to the AMD, COLAMD, and METIS
ordering methods, writing a matrix to a file, and many other
functions. ParU also relies on UMFPACK Version 6.0 or higher for symbolic 
analysis. 


%-------------------------------------------------------------------------------
\subsection{Installing the C/C++ library on Linux/Unix}
%-------------------------------------------------------------------------------

Before you compile the ParU library and demo programs, you may wish to
edit the 

\verb'SuiteSparse/SuiteSparse_config/SuiteSparse_config.mk' 
configuration file.  The defaults should be fine on most Linux/Unix systems and 
on the Mac.
It automatically detects what system you have and sets compile parameters
accordingly.

The configuration file defines where the LAPACK and BLAS libraries are to be
found.  Selecting the right BLAS is critical.  There is no standard naming
scheme for the name and location of these libraries.  The defaults in the
\verb'SuiteSparse_config.mk' file use \verb'-llapack' and \verb'-lblas';
the latter may link against the standard Fortran reference BLAS, which will not 
provide optimal performance.  For best results, you should use the OpenBLAS
at openblas.net
(based on the Goto BLAS)
\cite{GotoVanDeGeijn08}, or high-performance vendor-supplied BLAS such as the
Intel MKL, AMD ACML, or the Sun Performance Library.  Selection of LAPACK and
the BLAS is done with the \verb'LAPACK=' and \verb'BLAS=' lines in the
\verb'SuiteSparse_config.mk' file.

There are two parts that are important in chosing the compiler and 
\verb'BLAS' library.


\verb 'AUTOCC ?= yes' This line let \verb'SuiteSparse_config' choose the 
compiler automitcally. If there is intel compiler avialable it will be chose. 
If you change \verb'yes' to \verb'no' then GCC will be used for the compilation.


\verb 'BLAS ?= -lopenblas' This line let \verb'SuiteSparse_config' choose the 
\verb'BLAS' library. By default ParU uses  \verb'openBLAS'. If you comment out
this line ParU will look for Intel Math Kernel library. 

After you decide about the compiler and \verb'BLAS' library, type \verb'make' at 
the Linux/Unix command line, in either the 
\verb'SuiteSparse' directory (which compiles all of SuiteSparse) or in the 
\verb'SuiteSparse/ParU' directory (which just compiles ParU and the 
libraries it requires)???.  ParU will be compiled, and a set of simple demos 
will be run (including the one in the next section).

To  test the lines of ParU, go to the \verb'Tcov'
directory and type \verb'make'.  To fully test 100\% of the lines of ParU you 
should define \verb'PARU_ALLOC_TESTING' and \verb'PARU_COVERAGE' in
\verb'ParU\Source\paru_internal.hpp'.
This will work for Linux only.

To install the shared library
into /usr/local/lib and /usr/local/include, do {\tt make install}.
To uninstall, do {\tt make uninstall}.
For more options, see the {\tt SuiteSparse/README.txt} file.

%-------------------------------------------------------------------------------
\subsection{C/C++ Example}
%-------------------------------------------------------------------------------

The C++ interface is written using only real matrices.  
The simplest function computes the MATLAB equivalent of
\verb'x=A\b' and is almost as simple:
{\footnotesize
\begin{verbatim}
    #include "SuiteSparseQR.hpp"
    X = SuiteSparseQR <double> (A, B, cc) ;
\end{verbatim}
}
The C version of this function is almost identical:
{\footnotesize
\begin{verbatim}
    #include "SuiteSparseQR_C.h"
    X = SuiteSparseQR_C_backslash_default (A, B, cc) ;
\end{verbatim}
}

Below is a simple C++ program that illustrates the use of SuiteSparseQR.  The
program reads in a least-squares problem from \verb'stdin' in MatrixMarket
format \cite{BoisvertPozoRemingtonBarrettDongarra97}, solves it, and prints the
norm of the residual and the estimated rank of \verb'A'.  The comments reflect
the MATLAB equivalent statements.  The C version of this program is identical
except for the \verb'#include' statement and call to SuiteSparseQR which are
replaced with the C version of the statement above, and C-style comments.

\begin{minted}{c}
#include "ParU.hpp"

int main(int argc, char **argv)
{
    cholmod_common Common, *cc;
    cholmod_sparse *A;
    ParU_Symbolic *Sym = NULL;

    //~~~~~~~~~Reading the input matrix and test if the format is OK~~~~~~~~~~~~
    // start CHOLMOD
    cc = &Common;
    int mtype;
    cholmod_l_start(cc);

    // A = mread (stdin) ; read in the sparse matrix A
    A = (cholmod_sparse *)cholmod_l_read_matrix(stdin, 1, &mtype, cc);
    if (A == NULL)
    {
        printf("ParU: input matrix is invalid\n");
        exit(1);
    }

    if (mtype != CHOLMOD_SPARSE)
    {
        printf("ParU: input matrix must be sparse\n");
        exit(1);
    }

    if (A->xtype != CHOLMOD_REAL)
    {
        printf("ParU: input matrix must be real\n");
        exit(1);
    }

    //~~~~~~~~~~~~~~~~~~~Starting computation~~~~~~~~~~~~~~~~~~~~~~~~~~~~~~~~~~
    ParU_Control Control;
    ParU_Ret info;
    info = ParU_Analyze(A, &Sym, &Control);
    if (info != PARU_SUCCESS)
    {
        cholmod_l_free_sparse(&A, cc);
        cholmod_l_finish(cc);
        return info;
    }
    printf("In: %ldx%ld nnz = %ld \n", Sym->m, Sym->n, Sym->anz);
    printf("ParU: Symbolic factorization is done!\n");
    ParU_Numeric *Num;
    info = ParU_Factorize(A, Sym, &Num, &Control);
    double my_time = omp_get_wtime() - my_start_time;
    if (info != PARU_SUCCESS)
    {
        printf("ParU: factorization was NOT successfull in %lf seconds.\n",
               my_time);
        cholmod_l_free_sparse(&A, cc);
        cholmod_l_finish(cc);
        ParU_Freesym(&Sym, &Control);
        return info;
    }
    else
        printf("ParU: factorization was successfull in %lf seconds.\n",
               my_time);
    //~~~~~~~~~~~~~~~~~~~Test the results ~~~~~~~~~~~~~~~~~~~~~~~~~~~~~~~~~~~~
    Int m = Sym->m;
    if (info == PARU_SUCCESS)
    {
        double *b = (double *)malloc(m * sizeof(double));
        double *xx = (double *)malloc(m * sizeof(double));
        for (Int i = 0; i < m; ++i) b[i] = i + 1;
        info = ParU_Solve(Sym, Num, b, xx, &Control);
        if (info != PARU_SUCCESS)
        {
            printf("ParU: Solve has a problem.\n");
            free(b);
            free(xx);
            cholmod_l_free_sparse(&A, cc);
            cholmod_l_finish(cc);
            ParU_Freesym(&Sym, &Control);
            return info;
        }
        printf("Solve time is %lf seconds.\n", my_solve_time);
        double resid, anorm;
        info = ParU_Residual(A, xx, b, m, resid, anorm, &Control);
        if (info != PARU_SUCCESS)
        {
            printf("ParU: Residual has a problem.\n");
            free(b);
            free(xx);
            cholmod_l_free_sparse(&A, cc);
            cholmod_l_finish(cc);
            ParU_Freesym(&Sym, &Control);
            return info;
        }

        printf("Residual is |%.2lf| and anorm is %.2e and rcond is %.2e.\n",
                resid == 0 ? 0 : log10(resid), anorm, Num->rcond);

        free(b);
        free(xx);
            //~~~~~~~~~~~~~~~~~~~End computation~~~~~~~~~~~~~~~~~~~~~~~~~~~~~~~~~~~~~~
    Int max_threads = omp_get_max_threads();
    BLAS_set_num_threads(max_threads);

    //~~~~~~~~~~~~~~~~~~~Free Everything~~~~~~~~~~~~~~~~~~~~~~~~~~~~~~~~~~
    ParU_Freenum(&Num, &Control);
    ParU_Freesym(&Sym, &Control);

    cholmod_l_free_sparse(&A, cc);
    cholmod_l_finish(cc);
}

\end{minted}


{\footnotesize
\begin{verbatim}
    #include "SuiteSparseQR.hpp"
    int main (int argc, char **argv)
    {
        cholmod_common Common, *cc ;
        cholmod_sparse *A ;
        cholmod_dense *X, *B, *Residual ;
        double rnorm, one [2] = {1,0}, minusone [2] = {-1,0} ;
        int mtype ;

        // start CHOLMOD
        cc = &Common ;
        cholmod_l_start (cc) ;

        // load A
        A = (cholmod_sparse *) cholmod_l_read_matrix (stdin, 1, &mtype, cc) ;

        // B = ones (size (A,1),1)
        B = cholmod_l_ones (A->nrow, 1, A->xtype, cc) ;

        // X = A\B
        X = SuiteSparseQR <double> (A, B, cc) ;

        // rnorm = norm (B-A*X)
        Residual = cholmod_l_copy_dense (B, cc) ;
        cholmod_l_sdmult (A, 0, minusone, one, X, Residual, cc) ;
        rnorm = cholmod_l_norm_dense (Residual, 2, cc) ;
        printf ("2-norm of residual: %8.1e\n", rnorm) ;
        printf ("rank %ld\n", cc->SPQR_istat [4]) ;

        // free everything and finish CHOLMOD
        cholmod_l_free_dense (&Residual, cc) ;
        cholmod_l_free_sparse (&A, cc) ;
        cholmod_l_free_dense (&X, cc) ;
        cholmod_l_free_dense (&B, cc) ;
        cholmod_l_finish (cc) ;
        return (0) ;
    }
\end{verbatim}
}

%-------------------------------------------------------------------------------
\subsection{C++ Syntax}
%-------------------------------------------------------------------------------

All features available to the MATLAB user are also available to both the C and
C++ interfaces using a syntax that is not much more complicated than the MATLAB
syntax.  Additional features not available via the MATLAB interface include the
ability to compute the symbolic and numeric factorizations separately (for
multiple matrices with the same nonzero pattern but different numerical
values).  The following is a list of user-callable C++ functions and what they
can do:

\begin{enumerate}

    \item \verb'SuiteSparseQR': an overloaded function that provides functions
    equivalent to \verb'spqr' and \verb'spqr_solve' in the SuiteSparseQR MATLAB
    interface.

    \item \verb'SuiteSparseQR_factorize': performs both the symbolic and
    numeric factorizations and returns a QR factorization object such that
    \verb'A*P=Q*R'.  It always exploits singletons.

    \item \verb'SuiteSparseQR_symbolic': performs the symbolic factorization
    and returns a QR factorization object to be passed to
    \verb'SuiteSparseQR_numeric'.  It does not exploit singletons.

    \item \verb'SuiteSparseQR_numeric': performs the numeric factorization on a
    QR factorization object, either one constructed by
    \verb'SuiteSparseQR_symbolic', or reusing one from a prior call to
    \verb'SuiteSparseQR_numeric' for a matrix \verb'A' with the same pattern as
    the first one, but with different numerical values.

    \item \verb'SuiteSparseQR_solve': solves a linear system using the object
    returned by \newline \verb'SuiteSparseQR_factorize' or
    \verb'SuiteSparseQR_numeric', namely \verb"x=R\b", \newline \verb"x=P*R\b",
    \verb"x=R'\b", or \verb"x=R'\(P'*b)".

    \item \verb'SuiteSparseQR_qmult': provides the same function as
    \verb'spqr_qmult' in the MATLAB interface, computing
    \verb"Q'*x", \verb"Q*x", \verb"x*Q'", or \verb"x*Q".
    It uses either a QR factorization
    in MATLAB-style sparse matrix format, or the QR factorization object
    returned by \newline \verb'SuiteSparseQR_factorize' or
    \verb'SuiteSparseQR_numeric'.

    \item \verb'SuiteSparseQR_min2norm': finds the minimum 2-norm solution to
    an underdetermined linear system.

    \item \verb'SuiteSparseQR_free': frees the QR factorization object.

\end{enumerate}

%-------------------------------------------------------------------------------
\subsection{Details of the C/C++ Syntax}
%-------------------------------------------------------------------------------

For further details of how to use the C/C++ syntax, please refer to the
definitions and descriptions in the following files:

\begin{enumerate}
\item \verb'SuiteSparse/SPQR/Include/SuiteSparseQR.hpp' describes each
C++ function.  Both \verb'double' and \verb'std::complex<double>' matrices
are supported.

\item \verb'SuiteSparse/SPQR/Include/SuiteSparseQR_definitions.h' describes
definitions \newline common to both C and C++ functions.  For example, each of
the ordering methods is given a \verb'#define''d name.  The default is
\verb'ordering = SPQR_ORDERING_DEFAULT', and the default tolerance is given by
\verb'tol = SPQR_DEFAULT_TOL'.

\item \verb'SuiteSparse/SPQR/Include/SuiteSparseQR_C.h' describes
the C-callable functions.

\end{enumerate}

Most of the packages in SuiteSparse come in multiple versions with different
sized integers.  The first is the plain C/C++ \verb'int'.  The second the
\verb'SuiteSparse_long' integer, defined in the
\verb'SuiteSparse/SuiteSparse_config/SuiteSparse_config.h'
file.  This integer is \verb'long' except on a Windows-64 platform for which it
is the  \verb'__int64' type.  The intent of \verb'SuiteSparse_long'
is that it should be
32-bits on a 32-bit platform, and 64-bits on a 64-bit platform.  

By contrast, SuiteSparseQR only provides a \verb'SuiteSparse_long' version.
Most users (except Windows-64) can simply use \verb'long' as the basic integer
type passed to and returned from SuiteSparseQR.

The C/C++ options corresponding to the MATLAB \verb'opts' parameters and the
contents of the optional \verb'info' output of \verb'spqr_solve' are described
below.  Let \verb'cc' be the CHOLMOD \verb'Common' object, containing parameter
settings and statistics.  All are of type \verb'double', except for
\verb'SPQR_istat' which is \verb'SuiteSparse_long',
\verb'cc->memory_usage' which is
\verb'size_t', and \verb'cc->SPQR_nthreads' which is \verb'int'.  Parameters
include:

\vspace{0.1in}
{\footnotesize
\begin{tabular}{|ll|}
\hline
\verb'cc->SPQR_grain' & the same as \verb'opts.grain' in the MATLAB interface \\
\verb'cc->SPQR_small' & the same as \verb'opts.small' in the MATLAB interface \\
\verb'cc->SPQR_nthreads'
    & the same as \verb'opts.nthreads' in the MATLAB interface \\
\hline
\end{tabular}
}
\vspace{0.1in}

Other parameters, such as \verb'opts.ordering' and \verb'opts.tol',
are input parameters to the various C/C++ functions.  Others such as
\verb"opts.solution='min2norm'" are separate functions in the C/C++
interface.  Refer to the files listed above for details.
Output statistics include:

\vspace{0.1in}
{\footnotesize
\begin{tabular}{|ll|}
\hline
\verb'cc->SPQR_flopcount_bound' & an upper bound on the flop count \\
\verb'cc->SPQR_tol_used' & the tolerance used (\verb'opts.tol') \\
\hline
\verb'cc->SPQR_istat [0]' & upper bound on \verb'nnz(R)' \\
\verb'cc->SPQR_istat [1]' & upper bound on \verb'nnz(H)' \\
\verb'cc->SPQR_istat [2]' & number of frontal matrices \\
\verb'cc->SPQR_istat [3]' & number of TBB tasks \\
\verb'cc->SPQR_istat [4]' & estimate of the rank of \verb'A' \\
\verb'cc->SPQR_istat [5]' & number of column singletons \\
\verb'cc->SPQR_istat [6]' & number of row singletons \\
\verb'cc->SPQR_istat [7]' & ordering used \\
\hline
\verb'cc->memory_usage'   & memory used, in bytes \\
\hline
\end{tabular}
}
\vspace{0.1in}

The upper bound on the flop count is found in the analysis phase, which ignores
the numerical values of \verb'A' (the same analysis phase operates on both real
and complex matrices).  Thus, if you are factorizing a complex matrix, multiply
this statistic by 4.

%-------------------------------------------------------------------------------
\section{GPU acceleration}
\label{GPU}
%-------------------------------------------------------------------------------

As of version 2.0.0, SuiteSparseQR now includes GPU acceleration.
It can exploit a single NVIDIA GPU, via CUDA.  To enable GPU acceleration,
you must compile SuiteSparseQR with non-default options.  See the
\verb'SuiteSparse_config_GPU_gcc.mk' file in the \verb'SuiteSparse_config'
directory for details.  The packages SuiteSparse\_GPURuntime and
GPUQREngine are also required (they should appear in the SuiteSparse 
directory, along with SPQR).

At run time, you must also enable the GPU by setting \verb'Common->useGPU'
to \verb'true'.  Before calling any SuiteSparseQR function, you must
poll the GPU to set the available memory.  Below is a sample code
that initializes CHOLMOD and then polls the GPU for use in SuiteSparseQR.

\begin{verbatim}
    size_t total_mem, available_mem ;
    cholmod_common *cc, Common ;
    cc = &Common ;
    cholmod_l_start (cc) ;
    cc->useGPU = true ;
    cholmod_l_gpu_memorysize (&total_mem, &available_mem, cc) ;
    cc->gpuMemorySize = available_mem ;
    if (cc->gpuMemorySize <= 1)
    {
        printf ("no GPU available\n") ;
    }

    // Subsequent calls to SuiteSparseQR will use the GPU, if available
\end{verbatim}

See \verb'Demo/qrdemo_gpu.cpp' for an extended example, which can be
compiled via \verb'make gpu' in the \verb'Demo' directory.

GPU acceleration is not yet available via the MATLAB mexFunction interface. 
We expect to include this in a future release.

For a detailed technical report on the GPU-accelerated algorithm,
see \verb'qrgpu_paper.pdf' in the \verb'Doc' directory.

%-------------------------------------------------------------------------------
\section{Requirements and Availability}
\label{summary}
%-------------------------------------------------------------------------------

SuiteSparseQR requires four prior Collected Algorithms of the ACM: CHOLMOD
\cite{ChenDavisHagerRajamanickam09,DavisHager09} (version 1.7 or later), AMD
\cite{AmestoyDavisDuff96,AmestoyDavisDuff03}, and COLAMD
\cite{DavisGilbertLarimoreNg00_algo,DavisGilbertLarimoreNg00} for its
ordering/analysis phase and for its basic sparse matrix data structure, and the
BLAS \cite{dddh:90} for dense matrix computations on its frontal matrices; also
required is LAPACK \cite{LAPACK} for its Householder reflections.  An efficient
implementation of the BLAS is strongly recommended, either vendor-provided
(such as the Intel MKL, the AMD ACML, or the Sun Performance Library) or other
high-performance BLAS such as those of \cite{GotoVanDeGeijn08}.

The use of Intel's Threading Building Blocks is optional \cite{Reinders07}, but
without it, only parallelism within the BLAS can be exploited (if available).
Suite\-SparseQR can optionally use METIS 4.0.1 \cite{KarypisKumar98e} and two
constrained minimum degree ordering algorithms, CCOLAMD and CAMD
\cite{ChenDavisHagerRajamanickam09}, for its fill-reducing ordering options.
SuiteSparseQR can be compiled without these ordering methods and without TBB.

In addition to appearing as Collected Algorithm 8xx of the ACM, SuiteSparseQR
is available at
\htmladdnormallink{http://www.suitesparse.com}{http://www.suitesparse.com}
and at MATLAB Central
in the user-contributed File Exchange (
\htmladdnormallink{http://www.mathworks.com/matlabcentral}{http://www.mathworks.com/matlabcentral}
).
See SPQR/Doc/License.txt for the license.
Alternative licenses are also
available; contact the author for details.

%-------------------------------------------------------------------------------
% References
%-------------------------------------------------------------------------------

\bibliographystyle{plain}
\bibliography{paru_user_guide}
\end{document}

