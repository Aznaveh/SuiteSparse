\documentclass[12pt]{article}
\usepackage{hyperref}
\usepackage{minted}

\topmargin -0.5in
\textheight 9.0in
\oddsidemargin 0pt
\evensidemargin 0pt
\textwidth 6.5in

%-------------------------------------------------------------------------------
% get epsf.tex file, for encapsulated postscript files:
\input epsf
%-------------------------------------------------------------------------------
% macro for Postscript figures the easy way
% usage:  \postscript{file.ps}{scale}
% where scale is 1.0 for 100%, 0.5 for 50% reduction, etc.
%
\newcommand{\postscript}[2]
{\setlength{\epsfxsize}{#2\hsize}
\centerline{\epsfbox{#1}}}
%-------------------------------------------------------------------------------

\title{User's Guide for ParU, an unsymmetric multifrontal multithreaded sparse
LU factorization package}
\author{Mohsen Aznaveh\thanks{
email: aznaveh@tamu.edu.
http://www.suitesparse.com.
},
Timothy A. Davis}

\date{VERSION 0.0.0, May 20, 2022}

%-------------------------------------------------------------------------------
\begin{document}
%-------------------------------------------------------------------------------
\maketitle

\begin{abstract}

ParU is an implementation of the multifrontal sparse LU factorization
method.  Parallelism is exploited both in the BLAS and across different frontal
matrices using OpenMP tasking, a shared-memory programming model for modern 
multicore architectures. The package is written in C++ and real sparse matrices 
are supported.

\end{abstract}

\maketitle

%-------------------------------------------------------------------------------
\section{Introduction}
\label{intro}
%-------------------------------------------------------------------------------

The algorithms used in ParU will be discussed in a companion paper,
?. This document gives detailed information on the installation
and use of ParU.
ParU is a parallel sparse direct solver. This package uses OpenMP
tasking for parallelism. ParU calls UMFPACK for the symbolic analysis phase,
after that some symbolic analysis is done by ParU itself and  then the numeric
phase starts. The numeric computation is a task parallel phase using OpenMP
and each task calls parallel BLAS; i.e. nested parallelism. 
The performance of BLAS has a heavy impact on the performance of ParU.
However, depending on the input problem performance of parallelism in BLAS 
sometimes can have less effects in ParU.


%-------------------------------------------------------------------------------
\subsubsection{Instructions on using METIS}
%-------------------------------------------------------------------------------

SuiteSparse is now on METIS 5.1.0, which is distributed along with
SuiteSparse itself.  Its use is optional, however. ParU is using METIS as the 
default ordering. METIS tends to give orderings that are good for the 
parallelism. You can compile and run your code without using METIS; We recommend 
using METIS along with ParU.

Note that METIS is not bug-free; it can occasionally cause segmentation 
faults, particularly if used when finding basic solutions to underdetermined 
systems with many more columns than rows. ParU does not solve such 
systems anyway but you might see some problems with other SuiteSparse packages.

%-------------------------------------------------------------------------------
\section{Using ParU in C and C++}
%-------------------------------------------------------------------------------

ParU relies on CHOLMOD for its basic sparse matrix data structure, a compressed 
sparse column format.  CHOLMOD provides interfaces to the AMD, COLAMD, and METIS
ordering methods, and many other functions. ParU also relies on UMFPACK Version 
6.0 or higher for symbolic analysis. 


%-------------------------------------------------------------------------------
\subsection{Installing the C/C++ library on Linux/Unix}
%-------------------------------------------------------------------------------

Before you compile the ParU library and demo programs, you may wish to
edit the 

\verb'SuiteSparse/SuiteSparse_config/SuiteSparse_config.mk' 
configuration file.  The defaults should be fine on most Linux/Unix systems and 
on the Mac.
It automatically detects what system you have and sets compile parameters
accordingly.

The configuration file defines where the BLAS libraries are to be
found.  Selecting the right BLAS is critical.  There is no standard naming
scheme for the name and location of these libraries.  The defaults in the
\verb'SuiteSparse_config.mk' file use \verb'-lblas';
For best results, you should use the OpenBLAS at openblas.net (based on the 
Goto BLAS) \cite{GotoVanDeGeijn08}, or high-performance vendor-supplied BLAS 
such as the Intel MKL, AMD ACML, or the Sun Performance Library.  Selection of 
the BLAS is done with the \verb'BLAS=' lines in the
\verb'SuiteSparse_config.mk' file.

There are two parts that are important in chosing the compiler and the
\verb'BLAS' library.


\verb 'AUTOCC ?= yes' This line let \verb'SuiteSparse_config' choose the 
compiler automatically. If there is an Intel compiler available it will be chosen. 
If you change \verb'yes' to \verb'no' then GCC will be used for the compilation.


\verb 'BLAS ?= -lopenblas' This line let \verb'SuiteSparse_config' choose the 
\verb'BLAS' library. By default ParU uses  \verb'openBLAS'. If you comment out
this line ParU will look for the Intel Math Kernel Library. 

After you decide about the compiler and \verb'BLAS' library, type \verb'make' at 
the Linux/Unix command line, in either the 
\verb'SuiteSparse' directory (which compiles all of SuiteSparse) or in the 
\verb'SuiteSparse/ParU' directory (which just compiles ParU and the 
libraries it requires)???.  ParU will be compiled, and a set of simple demos 
will be run (including the one in the next section).

To  test the lines of ParU, go to the \verb'Tcov'
directory and type \verb'make'.  To fully test the lines of ParU you 
should define \verb'PARU_ALLOC_TESTING' and \verb'PARU_COVERAGE' in
\verb'ParU\Source\paru_internal.hpp'.
This will work for Linux only.

To install the shared library
into /usr/local/lib and /usr/local/include, do {\tt make install}.
To uninstall, do {\tt make uninstall}.
For more options, see the {\tt SuiteSparse/README.txt} file.

%-------------------------------------------------------------------------------
\subsection{C/C++ Example}
%-------------------------------------------------------------------------------

The C++ interface is written using only real matrices.  
The simplest function computes the MATLAB equivalent of
\verb'x=A\b' and is almost as simple:
Below is a simple C++ program that illustrates the use of ParU.  The
program reads in a problem from \verb'stdin' in MatrixMarket
format \cite{BoisvertPozoRemingtonBarrettDongarra97}, solves it, and prints the
norm of \verb'A' and the residual. 
Some error testing code is omited to simplify showing how the program works. 
The full program can be found in 
\verb'Paru/Demo/paru_demo.cpp'
\begin{minted}{c}
#include "ParU.hpp"
int main(int argc, char **argv)
{
    cholmod_common Common, *cc;
    cholmod_sparse *A;
    ParU_Symbolic *Sym = NULL;

    //~~~~~~~~~Reading the input matrix and test if the format is OK~~~~~~~~~~~~
    // start CHOLMOD
    cc = &Common;
    int mtype;
    cholmod_l_start(cc);

    // A = mread (stdin) ; read in the sparse matrix A
    A = (cholmod_sparse *)cholmod_l_read_matrix(stdin, 1, &mtype, cc);
    //~~~~~~~~~~~~~~~~~~~Starting computation~~~~~~~~~~~~~~~~~~~~~~~~~~~~~~~~~~
    ParU_Control Control;
    ParU_Ret info;
    info = ParU_Analyze(A, &Sym, &Control);
    ParU_Numeric *Num;
    info = ParU_Factorize(A, Sym, &Num, &Control);
    double my_time = omp_get_wtime() - my_start_time;
    //~~~~~~~~~~~~~~~~~~~Test the results ~~~~~~~~~~~~~~~~~~~~~~~~~~~~~~~~~~~~
    Int m = Sym->m;
    if (info == PARU_SUCCESS)
    {
        double *b = (double *)malloc(m * sizeof(double));
        double *xx = (double *)malloc(m * sizeof(double));
        for (Int i = 0; i < m; ++i) b[i] = i + 1;
        info = ParU_Solve(Sym, Num, b, xx, &Control);
        printf("Solve time is %lf seconds.\n", my_solve_time);
        double resid, anorm;
        info = ParU_Residual(A, xx, b, m, resid, anorm, &Control);
        printf("Residual is |%.2lf| and anorm is %.2e and rcond is %.2e.\n",
                resid == 0 ? 0 : log10(resid), anorm, Num->rcond);
        free(b);
        free(xx);
    }
    //~~~~~~~~~~~~~~~~~~~End computation~~~~~~~~~~~~~~~~~~~~~~~~~~~~~~~~~~~~~~
    Int max_threads = omp_get_max_threads();
    BLAS_set_num_threads(max_threads);

    //~~~~~~~~~~~~~~~~~~~Free Everything~~~~~~~~~~~~~~~~~~~~~~~~~~~~~~~~~~
    ParU_Freenum(&Num, &Control);
    ParU_Freesym(&Sym, &Control);

    cholmod_l_free_sparse(&A, cc);
    cholmod_l_finish(cc);
}
\end{minted}


%-------------------------------------------------------------------------------
\subsection{C/C++ Syntax}
%-------------------------------------------------------------------------------
\verb'ParU_Ret' is the output structure of all ParU routines. The user must 
check the output before continuing and computing further the result of prior
routine. You can see the user callable routines in 
\verb'Paru/Include/ParU.hpp'.
The following is a list of user-callable C++ functions and what they
can do:

\begin{enumerate}

    \item \verb'ParU_Version': return the version of the ParU package 
        you are using.

    \item \verb'ParU_Analyze': Symbolic analysis is done in this routine. 
        UMFPACK is called here and after that, some more specialized symbolic
        computation is done for ParU. 
        \verb'ParU_Analyze' called once and can be used for different 
        \verb'ParU_Factorize' calls for the matrices that have the same pattern.
    \item \verb'ParU_Factorize': 
        Numeric factorization is done in this routine. Scaling and
        making $Sx$ (scaled and stair case structure) matrix, computing factors 
        and permutations is here. \verb'ParU_Symbolic' structure which is 
        computed in \verb'ParU_Analyze' is an input in this routine.

    \item \verb'ParU_Solve':  
        Using symbolic analysis and factorization phase output to solve $Ax=b$.
        In all the solve routines Num structure must come with the same 
        Sym struct that comes from \verb'ParU_Factorize'. 
        This routine is overloaded and can solve different systems. It has 
        versions that keep a copy of x or overwrite it. Also, it can solve 
        multiple right-hand side problems.

    \item \verb'ParU_Freenum':  frees the numerical part of factorization.


    \item \verb'ParU_Freesym':  frees the symbolic part of factorization.

\end{enumerate}

%-------------------------------------------------------------------------------
\subsection{Details of the C/C++ Syntax}
%-------------------------------------------------------------------------------

For further details on how to use the C/C++ syntax, please refer to the
definitions and descriptions in the following files:

\begin{enumerate}
\item \verb'SuiteSparse/ParU/Include/ParU.hpp' describes each
C++ function.  Only \verb'double' and square matrices are supported.


\item \verb'SuiteSparse/ParU/Include/ParU.h' describes
the C-callable functions.

\end{enumerate}

There are C/C++ options to control ParU which is an input argument to several 
routines. When you make \verb'ParU_Control' object it is initialized with 
default values. The user can change the values. Here is the list of control 
options:

\vspace{0.1in}
{\footnotesize
\begin{tabular}{|lll|}
\hline
    \verb'ParU_Control' & default value & explanation  \\
\hline\hline
\verb'mem_chunk' & $1024*1024$ & chunk size for memset and memcpy\\
\verb'umfpack_ordering' & \verb'UMFPACK_ORDERING_METIS' & default UMFPACK ordering\\
\verb'umfpack_strategy' & \verb'UMFPACK_STRATEGY_AUTO'& default UMFPACK strategy\\
\verb'relaxed_amalgamation_threshold' & 32 & threshold for relaxed amalgamation \\
\hline
\verb'scale' & 1 & if 1 matrix will be scaled using \verb'max_row'\\
\verb'panel_width' & 32 & width of panel for dense factorizaiton\\
\verb'paru_strategy' & \verb'PARU_STRATEGY_AUTO' & default strategy for ParU\\
\verb'piv_toler' & $0.1$ & tolerance for accepting sparse pivots\\
\verb'diag_toler' & $0.001$ & tolerance for accepting symmetric pivots\\
\verb'trivial' & $4$ & Do not call BLAS for smaller dgemms\\
\verb'worthwhile_dgemm' & $512$ & dgemms bigger than worthwhile are tasked\\
\verb'worthwhile_trsm' & $4096$ & trsm bigger than worthwhile are tasked\\
\verb'paru_max_threads' & $0$ & initialized with \verb'omp_max_threads' \\
\hline
\end{tabular}
}
\vspace{0.1in}

The first row of the options is used in the symbolic analysis. In the symbolic 
analysis phase, only the pattern of the matrix are probed. 
The second row of control options show those that has an impact on numerical 
analysis.

\verb'paru_max_threads' is initalized by \verb'omp_max_threads' if the user do 
not provide a smaller number.

If \verb'paru_strategy' is set to \verb'PARU_STRATEGY_AUTO' ParU uses the same 
strategy as UMFPACK, however the user can ask UMFPACK for a unsymmetric 
strategy but use a symmetric strategy for ParU. Usually UMFPACK choses a good 
ordering, however there might be cases that user prefer unsymmetric ordering on
UMFPACK but symmetric computation on ParU.
    
%-------------------------------------------------------------------------------
\section{Requirements and Availability}
\label{summary}
%-------------------------------------------------------------------------------

ParU requires several Collected Algorithms of the ACM: CHOLMOD
\cite{ChenDavisHagerRajamanickam09,DavisHager09} (version 1.7 or later), AMD
\cite{AmestoyDavisDuff96,AmestoyDavisDuff03}, COLAMD
\cite{DavisGilbertLarimoreNg00_algo,DavisGilbertLarimoreNg00} and UMFPACK 
\cite{10.1145/992200.992206} for its
ordering/analysis phase and for its basic sparse matrix data structure, and the
BLAS \cite{dddh:90} for dense matrix computations on its frontal matrices. 
An efficient implementation of the BLAS is strongly recommended, either
vendor-provided (such as the Intel MKL, the AMD ACML, or the 
Sun Performance Library) or other high-performance BLAS such as those of 
\cite{GotoVanDeGeijn08}. Note that while ParU uses nested parallelism heavily
the right options for BLAS library must be chosen.

The use of OpenMP tasking is optional, but
without it, only parallelism within the BLAS can be exploited (if available).
See ParU/Doc/LICENSE for the license.
Alternative licenses are also
available; contact the author for details.

%-------------------------------------------------------------------------------
% References
%-------------------------------------------------------------------------------

\bibliographystyle{plain}
\bibliography{paru_user_guide}
\end{document}
